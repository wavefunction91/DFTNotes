\documentclass[11pt]{article}
\usepackage{amsmath, amsthm, amssymb, amsfonts, mathtools, xfrac, mathrsfs}
\usepackage[margin=0.5in]{geometry}

\newcommand{\opdm}[0]{\boldsymbol{\gamma}}
\newcommand{\pauli}[1]{\boldsymbol{\sigma}_{#1}}
\newcommand{\dd}[1]{\mathrm{d}{#1}\,}

\begin{document}

In density functioanl theory (DFT), the exchange-correlation (XC) energy, 
$E^{xc}$ is expressed as a functional of the single particle reduced density 
matrix, $\opdm$
\begin{equation}
E^{xc}[\opdm] = \int_{\mathbb{R}^3} \dd{r} f( V(r) ), \quad V(r) \equiv V(\opdm(r)),
\end{equation}
where $V$ is a set of fundamental density variables associated with
$\opdm$. The elements of $V$ depend on a number of factors related both
the spin structure of $\opdm$ and the nature of the approximations assumed
for $E^{xc}$, e.g. local density (LDA), generalized gradient (GGA) and 
meta-generalized gradient (MGGA) approximations. 
The spin structure of $\opdm$ is revealed through its realization as an hermitian rank-2 
complex tensor field on the spin manifold,
\begin{equation}
\opdm(r) = \begin{bmatrix} 
  \gamma^{\alpha\alpha}(r) & \gamma^{\alpha\beta}(r) \\ 
  \gamma^{\beta\alpha}(r) & \gamma^{\beta\beta}(r) 
\end{bmatrix}.
\end{equation}
Hermiticity dictates that $\gamma^{\alpha\alpha}, \gamma^{\beta\beta}$ are real valued while
only restricting that $\gamma^{\alpha\beta} = \bar{\gamma}^{\beta\alpha}$. Therefore, it is
often advantageous to express $\opdm$ in the basis of Pauli matrices, 
\begin{equation}
\opdm(r) = \frac{1}{2}n(r) \mathbf{I}_2 + \frac{1}{2}m_z(r) \pauli{z}
  + \frac{1}{2}m_y(r) \pauli{y}
  + \frac{1}{2}m_x(r) \pauli{x}
\end{equation}
where the Pauli matrices are defined as 
\begin{equation}
\pauli{z} = \begin{bmatrix} 1 & 0 \\ 0 & -1 \end{bmatrix},\quad
\pauli{y} = \begin{bmatrix} 0 & -i \\ i & 0 \end{bmatrix},\quad
\pauli{x} = \begin{bmatrix} 0 & 1  \\ 1 & 0 \end{bmatrix},
\end{equation}
and the transformed components are given by
\begin{align}
n(r)   &= \gamma^{\alpha\alpha} + \gamma^{\beta\beta},    \\ 
m_z(r) &= \gamma^{\alpha\alpha} - \gamma^{\beta\beta},    \\
m_y(r) &= i(\gamma^{\alpha\beta} - \gamma^{\beta\alpha}) = -2\Im \gamma^{\alpha\beta}, \\
m_x(r) &= \gamma^{\alpha\beta} + \gamma^{\beta\alpha}    = 2\Re \gamma^{\alpha\beta}.
\end{align}
The functions $n,m_x,m_y,m_z$ are all real valued and together completely 
describe $\opdm$. By restricting certain elements of this set to be zero,
we can impose various spin symmetries on our electronic wave function. In
particular, we consider the following three spin classifications
\begin{equation}
\opdm(r) \sim \begin{cases}
n(r), m_z(r), m_y(r), m_x(r) & \text{(GKS)} \\
n(r), m_z(r)                 & \text{(UKS)} \\
n(r)                         & \text{(RKS)}
\end{cases}
\end{equation}


In the local density approximation (LDA), $V^{LDA}(\opdm)$ is taken to only include
the non-zero density variables, e.g.
\begin{align}
V^{LDA}_{GKS} &= \{ n, m_z, m_y, m_x \} \nonumber \\
V^{LDA}_{UKS} &= \{ n, m_z \} \\
V^{LDA}_{RKS} &= \{ n \} \nonumber
\end{align}

In the generalized-gradient approximation (GGA), $V^{GGA}(\opdm)$ is taken the include
both the non-zero density variables and their gradients, e.g.
\begin{align}
V^{GGA}_{GKS} &= V^{LDA}_{GKS} \cup \{ \nabla n, \nabla m_z, \nabla m_y, \nabla m_x \} \nonumber \\
V^{GGA}_{UKS} &= V^{LDA}_{UKS} \cup \{ \nabla n, \nabla m_z \} \\
V^{GGA}_{RKS} &= V^{LDA}_{RKS} \cup \{ \nabla n \} \nonumber
\end{align}

As scalars such as the energy are invariant with respect to Galilean transformations, the orientation
dependence of the gradient variables required for e.g. GGA XC functional motivates the introduction 
of a set of varaiables which are invariant with respect to Galilean reference frame (i.e.
invariate with respect to spatial rotations, etc). As such, we untroduce a set of auxilary variables,
$U(\opdm)$ which take the following simple form for RKS
\begin{align}
U^{LDA}_{RKS}     &= V^{LDA}_{RKS} \\
U^{GGA}_{RKS}     &= U^{LDA}_{RKS} \cup \{ \eta \}
\end{align}
with $\eta = \| \nabla n \|^2$.
The expressions for the auxilarary variables for UKS/GKS take on a slightly more complicated form.
\begin{align}
U^{LDA}_{UKS/GKS} &= \{\rho_+, \rho_-\} \\
U^{GGA}_{UKS/GKS} &= U^{LDA}_{UKS/GKS} \cup \{ \eta_{++}, \eta_{+-}, \eta_{--}\}
\end{align}
Due to the fact that spin and spatial degrees of freedom are decoupled in UKS, these variables
take the following simple form
\begin{align}
&\rho_+^{UKS}    = \gamma^{\alpha\alpha}, \quad
\rho_-^{UKS}    = \gamma^{\beta\beta},   \\
&\eta_{++}^{UKS} = \|\nabla \gamma^{\alpha\alpha} \|^2, \quad
\eta_{+-}^{UKS} = \nabla \gamma^{\alpha\alpha} \cdot \nabla \gamma^{\beta\beta} , \quad
\eta_{--}^{UKS} = \|\nabla \gamma^{\beta\beta} \|^2. 
\end{align}
The coupling of spin and spatial degrees of freedom neccesitate the introduction of a more
complidated set of auxillary variables
\begin{align}
\rho_\pm^{GKS}      &= \frac{1}{2}n \pm \frac{1}{2}\sqrt{m_x^2 + m_y^2 + m_z^2}, \\
\eta_{\pm\pm}^{GKS} &= \frac{1}{2}\|\nabla n\|^2 + \frac{1}{2}\left( \sum_{i=x,y,z} \| \nabla m_i \|^2 \right) \pm 
                       \frac{\zeta_\nabla}{2} \sqrt{\sum_{i=x,y,z} \nabla n \cdot \nabla m_i}, \\
\eta_{+-}^{GKS}     &= \frac{1}{2}\|\nabla n\|^2 - \frac{1}{2}\left( \sum_{i=x,y,z} \| \nabla m_i \|^2 \right)  \\
\zeta_{\nabla}      &= \mathrm{sgn}\left( \sum_{i=x,y,z} (\nabla n \cdot \nabla m_i) m_i \right)
\end{align}

The introduction of the auxilarary variables neccesssitates the development of a kernel $g$ defined as
$g(U) = f(V)$, such that
\begin{equation}
E^{xc}[\opdm] = \int_{\mathbb{R}^3} \dd{r} g( U(r) ), \quad U(r) \equiv U(V(\opdm(r))),
\end{equation}

\section{Basis Set Expansions}

It is typically the case that on expands $\opdm$ in some basis $\{\chi_\mu(r)\}_{\mu = 1}^{N_b}$. This is typically achieved 
through the introduction of a density matrix, $\mathbf{D} \in \mathbb{C}^{2N_b\times 2N_b}$ which adopts an identical
spin structure to $\opdm$,
\begin{equation}
\mathbf{D} = \begin{bmatrix} 
  \mathbf{D}^{\alpha\alpha} & \mathbf{D}^{\alpha\beta} \\ 
  \mathbf{D}^{\beta\alpha} & \mathbf{D}^{\beta\beta} 
\end{bmatrix}, \quad
\mathbf{D}^{\sigma\sigma'} \in \mathbb{C}^{N_b\times N_b},\,\, \sigma,\sigma'\in\{\alpha,\beta\},
\end{equation}
such that
\begin{equation}
\gamma^{\sigma\sigma'}(r) = \sum_{\mu\nu} D_{\mu\nu}^{\sigma\sigma'} \chi_\mu(r) \chi_\nu(r).
\end{equation}
In particular, it is also possible to define analogous Pauli basis components for the density matrix
such that
\begin{equation}
\mathbf{D} = \frac{1}{2}\mathbf{N} \otimes \mathbf{I}_2 + 
\frac{1}{2}\mathbf{M}_z \otimes \pauli{z} + 
\frac{1}{2}\mathbf{M}_y \otimes \pauli{y} + 
\frac{1}{2}\mathbf{M}_x \otimes \pauli{x}, 
\end{equation}
where
\begin{align}
\mathbf{N}   &= \mathbf{D}^{\alpha\alpha} + \mathbf{D}^{\beta\beta},    \\ 
\mathbf{M}_z &= \mathbf{D}^{\alpha\alpha} - \mathbf{D}^{\beta\beta},    \\
\mathbf{M}_y &= -2\Im \mathbf{D}^{\alpha\beta}, \\
\mathbf{M}_x &= 2\Re \mathbf{D}^{\alpha\beta}.
\end{align}
such that
\begin{align}
n(r)   &= \sum_{\mu\nu} N_{\mu\nu} \chi_\mu(r) \chi_\nu(r), \\
m_i(r) &= \sum_{\mu\nu} (\mathbf{M}_i)_{\mu\nu} \chi_\mu(r) \chi_\nu(r), \quad i \in \{x,y,z\}. 
\end{align}

\section{Numerical Integration}

The nonlinear character of the integration kernel $f$ necessitates the evaluation of the XC energy (and associated
quantities such as its derivatives) via numerical integration. For molecular integrations, i.e. those with non-trivial
character in the vacinity of atomic nuclei, numerical integrations may be carried out via composite quadrature rule
\begin{equation}
E^{xc} \approx \sum_A \sum_{j \in \mathcal{Q}^A} w^A_j g(U(\opdm(r^A_j)))
\end{equation}
where $\mathcal{Q}^A$ is a spherical product quadrature centered at atomic center $A$. To avoid double-counting, the
quadrature weights of $\mathcal{Q}^A$ are modified by a partition function
\begin{equation}
w^A_j = p_A(r^A_j) w_j, \quad j \in \mathcal{Q}^A,
\end{equation}
where $w_j$ is an unmodified quadrature weight associated with the base quadrature of $\mathcal{Q}^A$. The partition function
$p_A$ generally depends on the entire molecular geometry, though domain decomposition in often possible.


\section{Gradients}

This change of variables $V\mapsto U$ necessacarily complicates the derivation of functional gradients
through the chain-rule.  Given an arbitrary perturbation $X$, we may express the XC gradient as
\begin{align}
\frac{\partial E^{xc}}{\partial X} &=
\sum_{IJ} \int_{\mathbb{R}^3} \dd{\mathbf{r}} \frac{\partial g(U)}{\partial U_I} \frac{\partial U_I}{\partial V_J} \frac{\partial V_J}{\partial X}, \\
&\approx \sum_A \sum_{j\in\mathcal{Q}^A} \sum_{IJ} \frac{\partial w_j^A}{\partial X} g(U(r_j^A)) + w_j^A\frac{\partial g(U(r_j^A))}{\partial U_I} \frac{\partial U_I(r_j^A)}{\partial V_J} \frac{\partial V_J(r^A_j)}{\partial X}
\end{align}
The sums over $I$ and $J$ run over the auxillary ($U$) and density ($V$) variables,
respectively. This decomposition breaks down the evaluation of the gradient into three contributions
\begin{enumerate}
  \item Derivatives of the XC kernel with respect to the auxillary variables ($\frac{\partial g}{\partial U_I}$). These are independent of perturbation and spin
        classification, and are typically handled by external libraries (Libxc, ExchCXX, etc) for point-wise evaluations. As such, we do not consider them here.
  \item The Jacobian for the variable transformation $V\mapsto U$ ($\mathcal{J}_{IJ} = \frac{\partial U_I}{\partial V_J}$). These are also independent of the
        perturbation but depend on the spin class being treated (RKS,UKS,GKS).
  \item Two perturbation dependent pieces ($\frac{\partial w_j}{\partial X}$, $\frac{\partial V_J}{\partial X}$) which depend both on the perturbation and
        spin classification. 
\end{enumerate}

\subsection{Auxillary Variable Jacobian}

\subsubsection{RKS}
LDA - trivial.

GGA:
\begin{align}
  &\frac{\partial \eta}{\partial (\nabla_i n)} = \frac{\partial}{\partial (\nabla_i n)} \nabla n \cdot \nabla n = 2 \nabla_i n
\end{align}

\subsubsection{UKS}
LDA:
\begin{align}
  &\frac{\partial \rho_\pm^{UKS}}{\partial n}   = \frac{1}{2} \frac{\partial}{\partial n} (n \pm m_z) = \frac{1}{2} \\ 
  &\frac{\partial \rho_\pm^{UKS}}{\partial m_z} = \frac{1}{2} \frac{\partial}{\partial m_z} (n \pm m_z) = \pm\frac{1}{2}  
\end{align}

GGA:
\begin{align}
  \frac{\partial \eta_{\pm\pm}}{\partial \nabla_i n} &= 
    \frac{1}{4} \frac{\partial}{\partial \nabla_i n}\left((\nabla n \pm \nabla m_z)\cdot(\nabla n \pm \nabla m_z)\right) \nonumber \\
  &= \frac{1}{4} \frac{\partial}{\partial \nabla_i n}\left(\nabla n\cdot \nabla n \pm 2 \nabla n\cdot \nabla m_z + \nabla m_z\cdot \nabla m_z\right) \nonumber \\
  &= \frac{1}{2}(\nabla_i n \pm \nabla_i m_z)\\
  \nonumber \\
  \frac{\partial \eta_{\pm\pm}}{\partial \nabla_i m_z} &= 
    \frac{1}{4} \frac{\partial}{\partial \nabla_i m_z}\left((\nabla n \pm \nabla m_z)\cdot(\nabla n \pm \nabla m_z)\right) \nonumber \\
  &= \frac{1}{4} \frac{\partial}{\partial \nabla_i m_z}\left(\nabla n\cdot \nabla n \pm 2 \nabla n\cdot \nabla m_z + \nabla m_z\cdot \nabla m_z\right) \nonumber \\
  &= \frac{1}{2}(\nabla_i m_z \pm \nabla_i n)\\
  \nonumber \\
  \frac{\partial \eta_{+-}}{\partial \nabla_i n} &= 
    \frac{1}{4} \frac{\partial}{\partial \nabla_i n}\left((\nabla n + \nabla m_z)\cdot(\nabla n - \nabla m_z)\right) \nonumber \\
  &= \frac{1}{4} \frac{\partial}{\partial \nabla_i n}\left(\nabla n\cdot \nabla n - \nabla m_z\cdot \nabla m_z\right) \nonumber \\
  &= \frac{1}{2} \nabla_i n \\
  \nonumber \\
  \frac{\partial \eta_{+-}}{\partial \nabla_i m_z} &= 
    \frac{1}{4} \frac{\partial}{\partial \nabla_i m_z}\left((\nabla n + \nabla m_z)\cdot(\nabla n - \nabla m_z)\right) \nonumber \\
  &= \frac{1}{4} \frac{\partial}{\partial \nabla_i m_z}\left(\nabla n\cdot \nabla n - \nabla m_z\cdot \nabla m_z\right) \nonumber \\
  &= -\frac{1}{2} \nabla_i m_z
\end{align}

\subsubsection{GKS}
TODO



\subsection{Exchange-Correlation Potentials}

Expressions for XC potentials may be derived as the gradient of the XC energy with respect to density matrix elements. Here we derive the primary 
perturbation dependent components and the final expressions

\subsubsection{RKS XC Potential}

Perturbation derivatives:
\begin{align}
&\frac{\partial n}{\partial N_{\lambda\kappa}} = \frac{\partial}{\partial N_{\lambda\kappa}} \sum_{\mu\nu} N_{\mu\nu} \chi_\mu \chi_\nu
  = \chi_\lambda \chi_\kappa \\ 
&\frac{\partial \nabla n}{\partial N_{\lambda\kappa}} = \frac{\partial}{\partial N_{\lambda\kappa}} \sum_{\mu\nu} N_{\mu\nu} \nabla(\chi_\mu \chi_\nu)
  = \nabla(\chi_\lambda \chi_\kappa) 
\end{align}

LDA:
\begin{align}
&\frac{\partial E^{xc}}{\partial N_{\mu\nu}} = \int_{\mathbb{R}^3} \dd{\mathbf{r}} \frac{\partial g(\mathbf{r})}{\partial n} \frac{\partial n}{\partial N_{\mu\nu}}
 = \int_{\mathbb{R}^3} \dd{\mathbf{r}} \frac{\partial g(\mathbf{r})}{\partial n} \chi_\mu(\mathbf{r}) \chi_\nu(\mathbf{r}) \\
&\frac{\partial E^{xc}}{\partial D^{\alpha\alpha}_{\mu\nu}} 
  = \frac{\partial E^{xc}}{\partial N_{\mu\nu}} \frac{\partial N_{\mu\nu}}{\partial D^{\alpha\alpha}_{\mu\nu}}
  = \frac{\partial E^{xc}}{\partial N_{\mu\nu}} 
\end{align}

GGA:
\begin{align}
\frac{\partial E^{xc}}{\partial N_{\mu\nu}} &= \int_{\mathbb{R}^3} \dd{\mathbf{r}} 
  \frac{\partial g(\mathbf{r})}{\partial n}   \frac{\partial n}   {\partial N_{\mu\nu}} +
  \frac{\partial g(\mathbf{r})}{\partial \eta}\frac{\partial \eta}{\partial \nabla n} \cdot\frac{\partial \nabla n}{\partial N_{\mu\nu}} \nonumber \\ 
&= \int_{\mathbb{R}^3} \dd{\mathbf{r}} \frac{\partial g(\mathbf{r})}{\partial n} \chi_\mu(\mathbf{r}) \chi_\nu(\mathbf{r}) +
   \frac{\partial g(\mathbf{r})}{\partial \eta} (2\nabla n) \cdot \nabla( \chi_\mu(\mathbf{r}) \chi_\nu(\mathbf{r}) ) \nonumber \\
&= \int_{\mathbb{R}^3} \dd{\mathbf{r}} \frac{\partial g(\mathbf{r})}{\partial n} \chi_\mu(\mathbf{r}) \chi_\nu(\mathbf{r}) +
   2\frac{\partial g(\mathbf{r})}{\partial \eta} \nabla n \cdot \nabla( \chi_\mu(\mathbf{r}) \chi_\nu(\mathbf{r}) )\\ 
\frac{\partial E^{xc}}{\partial D^{\alpha\alpha}_{\mu\nu}} 
  &= \frac{\partial E^{xc}}{\partial N_{\mu\nu}} \frac{\partial N_{\mu\nu}}{\partial D^{\alpha\alpha}_{\mu\nu}}
  = \frac{\partial E^{xc}}{\partial N_{\mu\nu}} 
\end{align}


\subsubsection{UKS XC Potential}
Perturbation derivatives:
\begin{align}
&\frac{\partial n}{\partial M^z_{\mu\nu}} = \frac{\partial \nabla n}{\partial M^z_{\mu\nu}} 
 = \frac{\partial m_z}{\partial N_{\mu\nu}} = \frac{\partial \nabla m_z}{\partial N_{\mu\nu}} = 0 \\
&\frac{\partial m_z}{\partial M^z_{\lambda\kappa}} = \frac{\partial}{\partial M^z_{\lambda\kappa}} \sum_{\mu\nu} M^z_{\mu\nu} \chi_\mu \chi_\nu
  = \chi_\lambda \chi_\kappa \\ 
&\frac{\partial \nabla m_z}{\partial M^z_{\lambda\kappa}} = \frac{\partial}{\partial M^z_{\lambda\kappa}} \sum_{\mu\nu} M^z_{\mu\nu} \nabla(\chi_\mu \chi_\nu)
  = \nabla(\chi_\lambda \chi_\kappa) 
\end{align}

LDA:
\begin{align}
\frac{\partial E^{xc}}{\partial N_{\mu\nu}} &= \int_{\mathbb{R}^3} \dd{\mathbf{r}} 
  \frac{\partial n}{\partial N_{\mu\nu}}\left( \frac{\partial g}{\partial \rho_+}\frac{\partial\rho_+}{\partial n} +
    \frac{\partial g}{\partial \rho_-}\frac{\partial\rho_-}{\partial n} \right) \nonumber \\
  &= \frac{1}{2}\int_{\mathbb{R}^3} \dd{\mathbf{r}} \left(\frac{\partial g}{\partial \rho_+} + \frac{\partial g}{\partial \rho_-} \right) \chi_\mu \chi_\nu \\
\frac{\partial E^{xc}}{\partial M^z_{\mu\nu}} &= \int_{\mathbb{R}^3} \dd{\mathbf{r}} 
  \frac{\partial m_z}{\partial M^z_{\mu\nu}}\left( \frac{\partial g}{\partial \rho_+}\frac{\partial\rho_+}{\partial m_z} +
    \frac{\partial g}{\partial \rho_-}\frac{\partial\rho_-}{\partial m_z} \right) \nonumber \\
  &= \frac{1}{2}\int_{\mathbb{R}^3} \dd{\mathbf{r}} \left(\frac{\partial g}{\partial \rho_+} - \frac{\partial g}{\partial \rho_-} \right) \chi_\mu \chi_\nu \\
\frac{\partial E^{xc}}{\partial D^{\alpha\alpha}_{\mu\nu}} &= 
  \frac{\partial E^{xc}}{\partial N_{\mu\nu}} \frac{\partial N_{\mu\nu}}{\partial D^{\alpha\alpha}_{\mu\nu}} +
  \frac{\partial E^{xc}}{\partial M^z_{\mu\nu}} \frac{\partial M^z_{\mu\nu}}{\partial D^{\alpha\alpha}_{\mu\nu}} = 
  \frac{\partial E^{xc}}{\partial N_{\mu\nu}} + \frac{\partial E^{xc}}{\partial M^z_{\mu\nu}}  \nonumber \\
  &= \int_{\mathbb{R}^3} \dd{\mathbf{r}} \frac{\partial g}{\partial \rho_+} \chi_\mu \chi_\nu \\
\frac{\partial E^{xc}}{\partial D^{\beta\beta}_{\mu\nu}} &= 
  \frac{\partial E^{xc}}{\partial N_{\mu\nu}} \frac{\partial N_{\mu\nu}}{\partial D^{\beta\beta}_{\mu\nu}} +
  \frac{\partial E^{xc}}{\partial M^z_{\mu\nu}} \frac{\partial M^z_{\mu\nu}}{\partial D^{\beta\beta}_{\mu\nu}} = 
  \frac{\partial E^{xc}}{\partial N_{\mu\nu}} - \frac{\partial E^{xc}}{\partial M^z_{\mu\nu}}  \nonumber \\
  &= \int_{\mathbb{R}^3} \dd{\mathbf{r}} \frac{\partial g}{\partial \rho_-} \chi_\mu \chi_\nu 
\end{align}

GGA:
\begin{align}
\frac{\partial E^{xc}}{\partial N_{\mu\nu}} &= \int_{\mathbb{R}^3} \dd{\mathbf{r}} 
  \frac{\partial n}{\partial N_{\mu\nu}}\left( 
      \frac{\partial g}{\partial \rho_+}\frac{\partial\rho_+}{\partial n} +
      \frac{\partial g}{\partial \rho_-}\frac{\partial\rho_-}{\partial n} 
    \right) +
  \frac{\partial \nabla n}{\partial N_{\mu\nu}}\cdot\left(
      \frac{\partial g}{\partial \eta_{++}}\frac{\partial\eta_{++}}{\partial \nabla n} +
      \frac{\partial g}{\partial \eta_{+-}}\frac{\partial\eta_{+-}}{\partial \nabla n} +
      \frac{\partial g}{\partial \eta_{--}}\frac{\partial\eta_{--}}{\partial \nabla n} 
    \right)\nonumber \\
  &= \frac{1}{2}\int_{\mathbb{R}^3} \dd{\mathbf{r}} 
    \left(\frac{\partial g}{\partial \rho_+} + \frac{\partial g}{\partial \rho_-} \right) \chi_\mu \chi_\nu +
    \left(
      \frac{\partial g}{\partial \eta_{++}}(\nabla n + \nabla m_z) +
      \frac{\partial g}{\partial \eta_{+-}}\nabla n +
      \frac{\partial g}{\partial \eta_{--}}(\nabla n - \nabla m_z) 
    \right) \cdot \nabla(\chi_\mu \chi_\nu ) \nonumber \\
  &= \frac{1}{2}\int_{\mathbb{R}^3} \dd{\mathbf{r}} 
    \left(\frac{\partial g}{\partial \rho_+} + \frac{\partial g}{\partial \rho_-} \right) \chi_\mu \chi_\nu +
    \left(
      \left[ \frac{\partial g}{\partial \eta_{++}} + \frac{\partial g}{\partial \eta_{+-}} + \frac{\partial g}{\partial \eta_{--}} \right] \nabla n +
      \left[ \frac{\partial g}{\partial \eta_{++}} - \frac{\partial g}{\partial \eta_{--}} \right] \nabla m_z
    \right) \cdot \nabla(\chi_\mu \chi_\nu ) \\
    \nonumber\\
    \nonumber\\
\frac{\partial E^{xc}}{\partial M^z_{\mu\nu}} &= \int_{\mathbb{R}^3} \dd{\mathbf{r}} 
  \frac{\partial m_z}{\partial M^z_{\mu\nu}}\left( 
      \frac{\partial g}{\partial \rho_+}\frac{\partial\rho_+}{\partial m_z} +
      \frac{\partial g}{\partial \rho_-}\frac{\partial\rho_-}{\partial m_z} 
    \right) +
  \frac{\partial \nabla n}{\partial M^z_{\mu\nu}}\cdot\left(
      \frac{\partial g}{\partial \eta_{++}}\frac{\partial\eta_{++}}{\partial \nabla m_z} +
      \frac{\partial g}{\partial \eta_{+-}}\frac{\partial\eta_{+-}}{\partial \nabla m_z} +
      \frac{\partial g}{\partial \eta_{--}}\frac{\partial\eta_{--}}{\partial \nabla m_z} 
    \right)\nonumber \\
  &= \frac{1}{2}\int_{\mathbb{R}^3} \dd{\mathbf{r}} 
    \left(\frac{\partial g}{\partial \rho_+} - \frac{\partial g}{\partial \rho_-} \right) \chi_\mu \chi_\nu +
    \left(
      \frac{\partial g}{\partial \eta_{++}}(\nabla n + \nabla m_z) -
      \frac{\partial g}{\partial \eta_{+-}}\nabla m_z -
      \frac{\partial g}{\partial \eta_{--}}(\nabla n - \nabla m_z) 
    \right) \cdot \nabla(\chi_\mu \chi_\nu ) \nonumber \\
  &= \frac{1}{2}\int_{\mathbb{R}^3} \dd{\mathbf{r}} 
    \left(\frac{\partial g}{\partial \rho_+} - \frac{\partial g}{\partial \rho_-} \right) \chi_\mu \chi_\nu +
    \left(
      \left[ \frac{\partial g}{\partial \eta_{++}} - \frac{\partial g}{\partial \eta_{--}} \right] \nabla n +
      \left[ \frac{\partial g}{\partial \eta_{++}} - \frac{\partial g}{\partial \eta_{+-}} + \frac{\partial g}{\partial \eta_{--}} \right] \nabla m_z
    \right) \cdot \nabla(\chi_\mu \chi_\nu ) \\
    \nonumber\\
    \nonumber\\
\frac{\partial E^{xc}}{\partial D^{\alpha\alpha}_{\mu\nu}} &= 
  \frac{\partial E^{xc}}{\partial N_{\mu\nu}} \frac{\partial N_{\mu\nu}}{\partial D^{\alpha\alpha}_{\mu\nu}} +
  \frac{\partial E^{xc}}{\partial M^z_{\mu\nu}} \frac{\partial M^z_{\mu\nu}}{\partial D^{\alpha\alpha}_{\mu\nu}} = 
  \frac{\partial E^{xc}}{\partial N_{\mu\nu}} + \frac{\partial E^{xc}}{\partial M^z_{\mu\nu}}  \nonumber \\
  &= \int_{\mathbb{R}^3} \dd{\mathbf{r}} \frac{\partial g}{\partial \rho_+} \chi_\mu \chi_\nu +
    \left( 
      \frac{\partial g}{\partial \eta_{++}} (\nabla n + \nabla m_z) + 
      \frac{1}{2}\frac{\partial g}{\partial \eta_{+-}} (\nabla n - \nabla m_z)
    \right) \cdot \nabla(\chi_\mu \chi_\nu ) \\
  &= \int_{\mathbb{R}^3} \dd{\mathbf{r}} \frac{\partial g}{\partial \rho_+} \chi_\mu \chi_\nu +
    \left( 
      2\frac{\partial g}{\partial \eta_{++}} \nabla \rho_+ + 
      \frac{\partial g}{\partial \eta_{+-}} \nabla \rho_- 
    \right) \cdot \nabla(\chi_\mu \chi_\nu ) \\
    \nonumber\\
    \nonumber\\
\frac{\partial E^{xc}}{\partial D^{\beta\beta}_{\mu\nu}} &= 
  \frac{\partial E^{xc}}{\partial N_{\mu\nu}} \frac{\partial N_{\mu\nu}}{\partial D^{\beta\beta}_{\mu\nu}} +
  \frac{\partial E^{xc}}{\partial M^z_{\mu\nu}} \frac{\partial M^z_{\mu\nu}}{\partial D^{\beta\beta}_{\mu\nu}} = 
  \frac{\partial E^{xc}}{\partial N_{\mu\nu}} - \frac{\partial E^{xc}}{\partial M^z_{\mu\nu}}  \nonumber \\
  &= \int_{\mathbb{R}^3} \dd{\mathbf{r}} \frac{\partial g}{\partial \rho_-} \chi_\mu \chi_\nu +
    \left( 
      \frac{\partial g}{\partial \eta_{--}} (\nabla n - \nabla m_z) + 
      \frac{1}{2}\frac{\partial g}{\partial \eta_{+-}} (\nabla n + \nabla m_z)
    \right) \cdot \nabla(\chi_\mu \chi_\nu ) \\
  &= \int_{\mathbb{R}^3} \dd{\mathbf{r}} \frac{\partial g}{\partial \rho_-} \chi_\mu \chi_\nu +
    \left( 
      2\frac{\partial g}{\partial \eta_{--}} \nabla \rho_- + 
      \frac{\partial g}{\partial \eta_{+-}} \nabla \rho_+ 
    \right) \cdot \nabla(\chi_\mu \chi_\nu ) 
\end{align}




\newpage

\subsection{Nuclear Gradients}

\subsubsection{RKS}
Let
\begin{align}
&Z^{N}_{\mu}  = \sum_\nu N_{\mu\nu} \chi_\nu \\
&Z^{Nj}_{\mu} = \sum_\nu N_{\mu\nu} (\nabla_j \chi_\nu) 
\end{align}
Perturbation derivatives:
\begin{align}
\frac{\partial n}{\partial R_{A_i}} &= \sum_{\mu\nu} N_{\mu\nu} \frac{\partial}{\partial R_{A_i}} (\chi_\mu \chi_\nu) = 
  2\sum_{\mu\nu} N_{\mu\nu} \frac{\partial \chi_\mu}{\partial R_{A_i}} \chi_\nu = 
  -2\sum_{\mu\in A} \sum_\nu N_{\mu\nu} (\nabla_i \chi_\mu) \chi_\nu \nonumber \\ &= 
  -2\sum_{\mu\in A} (\nabla_i \chi_\mu) Z^N_\mu \\
\nonumber \\
\frac{\partial \nabla_j n}{\partial R_{A_i}} &= 
  \sum_{\mu\nu} N_{\mu\nu} \frac{\partial}{\partial R_{A_i}} \nabla_j (\chi_\mu \chi_\nu) = 
  \sum_{\mu\nu} N_{\mu\nu} \nabla_j \frac{\partial}{\partial R_{A_i}} (\chi_\mu \chi_\nu) = 
  2\sum_{\mu\nu} N_{\mu\nu} \nabla_j \left(\frac{\partial \chi_\mu}{\partial R_{A_i}} \chi_\nu \right) \nonumber \\
  &= -2\sum_{\mu\in A} \sum_\nu N_{\mu\nu} \nabla_j \left((\nabla_i\chi_\mu) \chi_\nu \right) \nonumber \\
  &= -2\sum_{\mu\in A} \sum_\nu N_{\mu\nu} \left( (\nabla^{(2)}_{ij}\chi_\mu)\chi_\nu + (\nabla_i \chi_\mu)(\nabla_j \chi_\nu)\right) \nonumber \\
  &= -2\sum_{\mu\in A} \left( (\nabla^{(2)}_{ij}\chi_\mu) Z^{N}_\mu + (\nabla_i \chi_\mu)Z^{Nj}_\mu\right) 
\end{align}

LDA:
\begin{align}
\frac{\partial E^{xc}}{\partial R_{A_i}} = \int_{\mathbb{R}^3} \dd{\mathbf{r}} \frac{\partial g(\mathbf{r})}{\partial n} \frac{\partial n}{\partial R_{A_i}}
 = -2\sum_{\mu\in A}
    \int_{\mathbb{R}^3} \dd{\mathbf{r}} \frac{\partial g(\mathbf{r})}{\partial n} (\nabla_i \chi_\mu) Z^N_\mu
\end{align}

GGA:
\begin{align}
\frac{\partial E^{xc}}{\partial R_{A_i}} &= \int_{\mathbb{R}^3} \dd{\mathbf{r}} 
  \frac{\partial g(\mathbf{r})}{\partial n}   \frac{\partial n}   {\partial R_{A_i}} +
  \frac{\partial g(\mathbf{r})}{\partial \eta}\frac{\partial \eta}{\partial \nabla n} \cdot\frac{\partial \nabla n}{\partial R_{A_i}} \nonumber \\ 
&= -2\sum_{\mu\in A} 
   \int_{\mathbb{R}^3} \dd{\mathbf{r}} \frac{\partial g(\mathbf{r})}{\partial n} (\nabla_i \chi_\mu) Z^N_\mu +
   2\frac{\partial g(\mathbf{r})}{\partial \eta} \sum_j\nabla_j n \left( (\nabla^{(2)}_{ij}\chi_\mu) Z^N_\mu + (\nabla_i \chi_\mu) Z^{Nj}_\mu \right)
\end{align}


\subsubsection{UKS}
Perturbation derivatives:
\begin{align}
\frac{\partial m_z}{\partial R_{A_i}} &= -2\sum_{\mu\in A} \sum_\nu M^z_{\mu\nu} (\nabla_i \chi_\mu) \chi_\nu \\
\frac{\partial \nabla_j m_z}{\partial R_{A_i}} &= 
  -2\sum_{\mu\in A} \sum_\nu M^z_{\mu\nu} \left( (\nabla^{(2)}_{ij}\chi_\mu)\chi_\nu + (\nabla_i \chi_\mu)(\nabla_j \chi_\nu)\right) 
\end{align}

LDA:
\begin{align}
\frac{\partial E^{xc}}{\partial R_{A_i}} &= \int_{\mathbb{R}^3} \dd{\mathbf{r}} 
  \frac{\partial n}{\partial R_{A_i}}\left( \frac{\partial g}{\partial \rho_+}\frac{\partial\rho_+}{\partial n} +
    \frac{\partial g}{\partial \rho_-}\frac{\partial\rho_-}{\partial n} \right) +
  \frac{\partial m_z}{\partial R_{A_i}}\left( \frac{\partial g}{\partial \rho_+}\frac{\partial\rho_+}{\partial m_z} +
    \frac{\partial g}{\partial \rho_-}\frac{\partial\rho_-}{\partial m_z} \right) 
  \nonumber \\
&= \frac{1}{2}\int_{\mathbb{R}^3} \dd{\mathbf{r}} 
  \frac{\partial n}  {\partial R_{A_i}}\left( \frac{\partial g}{\partial \rho_+} + \frac{\partial g}{\partial \rho_-} \right) +
  \frac{\partial m_z}{\partial R_{A_i}}\left( \frac{\partial g}{\partial \rho_+} - \frac{\partial g}{\partial \rho_-} \right) 
  \nonumber \\
&= - \sum_{\mu\in A} \sum_\nu \int_{\mathbb{R}^3} \dd{\mathbf{r}}\left[ 
  N_{\mu\nu}  \left( \frac{\partial g}{\partial \rho_+} + \frac{\partial g}{\partial \rho_-} \right) +
  M^z_{\mu\nu}\left( \frac{\partial g}{\partial \rho_+} - \frac{\partial g}{\partial \rho_-} \right) 
  \right] (\nabla_i \chi_\mu) \chi_\nu \\
&= - 2\sum_{\mu\in A} \sum_\nu \int_{\mathbb{R}^3} \dd{\mathbf{r}}\left[ 
  D^{\alpha\alpha}_{\mu\nu} \frac{\partial g}{\partial \rho_+} +
  D^{\beta\beta}_{\mu\nu}   \frac{\partial g}{\partial \rho_-} 
  \right] (\nabla_i \chi_\mu) \chi_\nu
\end{align}

GGA:
\begin{align}
\frac{\partial E^{xc}}{\partial R_{A_i}} &= \int_{\mathbb{R}^3} \dd{\mathbf{r}} 
  \frac{\partial n}{\partial R_{A_i}}\left( \frac{\partial g}{\partial \rho_+}\frac{\partial\rho_+}{\partial n} +
    \frac{\partial g}{\partial \rho_-}\frac{\partial\rho_-}{\partial n} \right) +
  \frac{\partial m_z}{\partial R_{A_i}}\left( \frac{\partial g}{\partial \rho_+}\frac{\partial\rho_+}{\partial m_z} +
    \frac{\partial g}{\partial \rho_-}\frac{\partial\rho_-}{\partial m_z} \right) + 
  \nonumber \\ &\qquad\quad
  \frac{\partial \nabla n}{\partial R_{A_i}} \cdot \left( 
    \frac{\partial g}{\partial \eta_{++}}\frac{\partial\rho_+}{\partial \nabla n} +
    \frac{\partial g}{\partial \eta_{+-}}\frac{\partial\rho_-}{\partial \nabla n} +
    \frac{\partial g}{\partial \eta_{--}}\frac{\partial\rho_+}{\partial \nabla n} 
  \right) +
  \nonumber \\ &\qquad\quad
  \frac{\partial \nabla m_z}{\partial R_{A_i}} \cdot \left( 
    \frac{\partial g}{\partial \eta_{++}}\frac{\partial\rho_+}{\partial \nabla m_z} +
    \frac{\partial g}{\partial \eta_{+-}}\frac{\partial\rho_-}{\partial \nabla m_z} +
    \frac{\partial g}{\partial \eta_{--}}\frac{\partial\rho_+}{\partial \nabla m_z} 
  \right) 
  \nonumber \\
&= \frac{1}{2}\int_{\mathbb{R}^3} \dd{\mathbf{r}} 
  \frac{\partial n}  {\partial R_{A_i}}\left( \frac{\partial g}{\partial \rho_+} + \frac{\partial g}{\partial \rho_-} \right) +
  \frac{\partial m_z}{\partial R_{A_i}}\left( \frac{\partial g}{\partial \rho_+} - \frac{\partial g}{\partial \rho_-} \right) +
  \nonumber \\ &\qquad\quad
  \frac{\partial \nabla n}{\partial R_{A_i}} \cdot \left( 
      \left[ \frac{\partial g}{\partial \eta_{++}} + \frac{\partial g}{\partial \eta_{+-}} + \frac{\partial g}{\partial \eta_{--}} \right] \nabla n +
      \left[ \frac{\partial g}{\partial \eta_{++}} - \frac{\partial g}{\partial \eta_{--}} \right] \nabla m_z
  \right) +
  \nonumber \\ &\qquad\quad
  \frac{\partial \nabla m_z}{\partial R_{A_i}} \cdot \left( 
      \left[ \frac{\partial g}{\partial \eta_{++}} - \frac{\partial g}{\partial \eta_{--}} \right] \nabla n +
      \left[ \frac{\partial g}{\partial \eta_{++}} - \frac{\partial g}{\partial \eta_{+-}} + \frac{\partial g}{\partial \eta_{--}} \right] \nabla m_z
  \right) \nonumber\\ 
&= -\sum_{\mu\in A} \sum_\nu \int_{\mathbb{R}^3} \dd{\mathbf{r}} 
  \left[
  N_{\mu\nu} \left( \frac{\partial g}{\partial \rho_+} + \frac{\partial g}{\partial \rho_-} \right) +
  M^z_{\mu\nu}\left( \frac{\partial g}{\partial \rho_+} - \frac{\partial g}{\partial \rho_-} \right) 
  \right] (\nabla_i \chi_\mu) \chi_\nu +
  \nonumber \\ &\qquad\quad
  \sum_j \left\lbrace
    N_{\mu\nu}
    \left( 
      \left[ \frac{\partial g}{\partial \eta_{++}} + \frac{\partial g}{\partial \eta_{+-}} + \frac{\partial g}{\partial \eta_{--}} \right] \nabla_j n +
      \left[ \frac{\partial g}{\partial \eta_{++}} - \frac{\partial g}{\partial \eta_{--}} \right] \nabla_j m_z
    \right) \right. +
  \nonumber \\ &\qquad\quad
    M^z_{\mu\nu}
    \left.
    \left( 
        \left[ \frac{\partial g}{\partial \eta_{++}} - \frac{\partial g}{\partial \eta_{--}} \right] \nabla_j n +
        \left[ \frac{\partial g}{\partial \eta_{++}} - \frac{\partial g}{\partial \eta_{+-}} + \frac{\partial g}{\partial \eta_{--}} \right] \nabla_j m_z
    \right) 
  \right\rbrace
  \left( (\nabla^{(2)}_{ij}\chi_\mu)\chi_\nu + (\nabla_i \chi_\mu)(\nabla_j \chi_\nu)\right) 
\end{align}







\begin{align}
\rho(r_A) = \sum_{BC} \sum_{\mu\in B, \nu \in C}
  P_{\mu\nu} \chi_\mu( R_B, r_A ) \chi_\nu (R_C, r_A)
\end{align}
\begin{align}
\frac{\partial \rho(r_A)}{\partial R_C} &= \sum_{BC} \sum_{\mu\in B, \nu \in C}
  P_{\mu\nu} \frac{\partial}{\partial R_D}(\chi_\mu( R_B, r_A ) \chi_\nu (R_C, r_A)) \\
&= 
\sum_{BC} \sum_{\mu\in B, \nu \in C}
  P_{\mu\nu} \frac{\partial \chi_\mu( R_B, r_A )}{\partial R_D} \chi_\nu (R_C, r_A) + 
\sum_{BC} \sum_{\mu\in B, \nu \in C}
  P_{\mu\nu} \chi_\mu (R_B, r_A) \frac{\partial \chi_\nu( R_C, r_A )}{\partial R_D} \\ 
&= 
\sum_{BC} \sum_{\mu\in B, \nu \in C}
  P_{\mu\nu} \frac{\partial \chi_\mu( R_B, r_A )}{\partial R_D} \chi_\nu (R_C, r_A) + 
\sum_{BC} \sum_{\mu\in C, \nu \in B}
  P_{\mu\nu} \chi_\nu (R_B, r_A) \frac{\partial \chi_\mu( R_B, r_A )}{\partial R_D} \\ 
\end{align}



\end{document}
